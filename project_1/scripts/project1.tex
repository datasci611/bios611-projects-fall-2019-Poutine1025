\documentclass[]{article}
\usepackage{lmodern}
\usepackage{amssymb,amsmath}
\usepackage{ifxetex,ifluatex}
\usepackage{fixltx2e} % provides \textsubscript
\ifnum 0\ifxetex 1\fi\ifluatex 1\fi=0 % if pdftex
  \usepackage[T1]{fontenc}
  \usepackage[utf8]{inputenc}
\else % if luatex or xelatex
  \ifxetex
    \usepackage{mathspec}
  \else
    \usepackage{fontspec}
  \fi
  \defaultfontfeatures{Ligatures=TeX,Scale=MatchLowercase}
\fi
% use upquote if available, for straight quotes in verbatim environments
\IfFileExists{upquote.sty}{\usepackage{upquote}}{}
% use microtype if available
\IfFileExists{microtype.sty}{%
\usepackage{microtype}
\UseMicrotypeSet[protrusion]{basicmath} % disable protrusion for tt fonts
}{}
\usepackage[margin=1in]{geometry}
\usepackage{hyperref}
\hypersetup{unicode=true,
            pdftitle={Project 1: Urban Ministries of Durham},
            pdfauthor={Yangjianchen Xu},
            pdfborder={0 0 0},
            breaklinks=true}
\urlstyle{same}  % don't use monospace font for urls
\usepackage{color}
\usepackage{fancyvrb}
\newcommand{\VerbBar}{|}
\newcommand{\VERB}{\Verb[commandchars=\\\{\}]}
\DefineVerbatimEnvironment{Highlighting}{Verbatim}{commandchars=\\\{\}}
% Add ',fontsize=\small' for more characters per line
\usepackage{framed}
\definecolor{shadecolor}{RGB}{248,248,248}
\newenvironment{Shaded}{\begin{snugshade}}{\end{snugshade}}
\newcommand{\AlertTok}[1]{\textcolor[rgb]{0.94,0.16,0.16}{#1}}
\newcommand{\AnnotationTok}[1]{\textcolor[rgb]{0.56,0.35,0.01}{\textbf{\textit{#1}}}}
\newcommand{\AttributeTok}[1]{\textcolor[rgb]{0.77,0.63,0.00}{#1}}
\newcommand{\BaseNTok}[1]{\textcolor[rgb]{0.00,0.00,0.81}{#1}}
\newcommand{\BuiltInTok}[1]{#1}
\newcommand{\CharTok}[1]{\textcolor[rgb]{0.31,0.60,0.02}{#1}}
\newcommand{\CommentTok}[1]{\textcolor[rgb]{0.56,0.35,0.01}{\textit{#1}}}
\newcommand{\CommentVarTok}[1]{\textcolor[rgb]{0.56,0.35,0.01}{\textbf{\textit{#1}}}}
\newcommand{\ConstantTok}[1]{\textcolor[rgb]{0.00,0.00,0.00}{#1}}
\newcommand{\ControlFlowTok}[1]{\textcolor[rgb]{0.13,0.29,0.53}{\textbf{#1}}}
\newcommand{\DataTypeTok}[1]{\textcolor[rgb]{0.13,0.29,0.53}{#1}}
\newcommand{\DecValTok}[1]{\textcolor[rgb]{0.00,0.00,0.81}{#1}}
\newcommand{\DocumentationTok}[1]{\textcolor[rgb]{0.56,0.35,0.01}{\textbf{\textit{#1}}}}
\newcommand{\ErrorTok}[1]{\textcolor[rgb]{0.64,0.00,0.00}{\textbf{#1}}}
\newcommand{\ExtensionTok}[1]{#1}
\newcommand{\FloatTok}[1]{\textcolor[rgb]{0.00,0.00,0.81}{#1}}
\newcommand{\FunctionTok}[1]{\textcolor[rgb]{0.00,0.00,0.00}{#1}}
\newcommand{\ImportTok}[1]{#1}
\newcommand{\InformationTok}[1]{\textcolor[rgb]{0.56,0.35,0.01}{\textbf{\textit{#1}}}}
\newcommand{\KeywordTok}[1]{\textcolor[rgb]{0.13,0.29,0.53}{\textbf{#1}}}
\newcommand{\NormalTok}[1]{#1}
\newcommand{\OperatorTok}[1]{\textcolor[rgb]{0.81,0.36,0.00}{\textbf{#1}}}
\newcommand{\OtherTok}[1]{\textcolor[rgb]{0.56,0.35,0.01}{#1}}
\newcommand{\PreprocessorTok}[1]{\textcolor[rgb]{0.56,0.35,0.01}{\textit{#1}}}
\newcommand{\RegionMarkerTok}[1]{#1}
\newcommand{\SpecialCharTok}[1]{\textcolor[rgb]{0.00,0.00,0.00}{#1}}
\newcommand{\SpecialStringTok}[1]{\textcolor[rgb]{0.31,0.60,0.02}{#1}}
\newcommand{\StringTok}[1]{\textcolor[rgb]{0.31,0.60,0.02}{#1}}
\newcommand{\VariableTok}[1]{\textcolor[rgb]{0.00,0.00,0.00}{#1}}
\newcommand{\VerbatimStringTok}[1]{\textcolor[rgb]{0.31,0.60,0.02}{#1}}
\newcommand{\WarningTok}[1]{\textcolor[rgb]{0.56,0.35,0.01}{\textbf{\textit{#1}}}}
\usepackage{graphicx,grffile}
\makeatletter
\def\maxwidth{\ifdim\Gin@nat@width>\linewidth\linewidth\else\Gin@nat@width\fi}
\def\maxheight{\ifdim\Gin@nat@height>\textheight\textheight\else\Gin@nat@height\fi}
\makeatother
% Scale images if necessary, so that they will not overflow the page
% margins by default, and it is still possible to overwrite the defaults
% using explicit options in \includegraphics[width, height, ...]{}
\setkeys{Gin}{width=\maxwidth,height=\maxheight,keepaspectratio}
\IfFileExists{parskip.sty}{%
\usepackage{parskip}
}{% else
\setlength{\parindent}{0pt}
\setlength{\parskip}{6pt plus 2pt minus 1pt}
}
\setlength{\emergencystretch}{3em}  % prevent overfull lines
\providecommand{\tightlist}{%
  \setlength{\itemsep}{0pt}\setlength{\parskip}{0pt}}
\setcounter{secnumdepth}{0}
% Redefines (sub)paragraphs to behave more like sections
\ifx\paragraph\undefined\else
\let\oldparagraph\paragraph
\renewcommand{\paragraph}[1]{\oldparagraph{#1}\mbox{}}
\fi
\ifx\subparagraph\undefined\else
\let\oldsubparagraph\subparagraph
\renewcommand{\subparagraph}[1]{\oldsubparagraph{#1}\mbox{}}
\fi

%%% Use protect on footnotes to avoid problems with footnotes in titles
\let\rmarkdownfootnote\footnote%
\def\footnote{\protect\rmarkdownfootnote}

%%% Change title format to be more compact
\usepackage{titling}

% Create subtitle command for use in maketitle
\providecommand{\subtitle}[1]{
  \posttitle{
    \begin{center}\large#1\end{center}
    }
}

\setlength{\droptitle}{-2em}

  \title{Project 1: Urban Ministries of Durham}
    \pretitle{\vspace{\droptitle}\centering\huge}
  \posttitle{\par}
    \author{Yangjianchen Xu}
    \preauthor{\centering\large\emph}
  \postauthor{\par}
      \predate{\centering\large\emph}
  \postdate{\par}
    \date{9/24/2019}


\begin{document}
\maketitle

\begin{Shaded}
\begin{Highlighting}[]
\KeywordTok{library}\NormalTok{(tidyverse)}
\KeywordTok{library}\NormalTok{(readr)}
\KeywordTok{library}\NormalTok{(lubridate)}
\end{Highlighting}
\end{Shaded}

\hypertarget{background}{%
\subsection{Background}\label{background}}

The data is provided by Urban Ministries of Durham, of which the people
in this organization connect with the community to end homelessness and
fight poverty by offering food, shelter and a future to neighbors in
need. The data includes 79838 observations and 18 variables.

\hypertarget{data-import}{%
\subsection{Data import}\label{data-import}}

\begin{Shaded}
\begin{Highlighting}[]
\NormalTok{UMD=}\KeywordTok{read_tsv}\NormalTok{(}\StringTok{"~/Documents/GitHub/bios611-projects-fall-2019-Poutine1025/project_1/data/UMD_Services_Provided_20190719.tsv"}\NormalTok{)}
\NormalTok{metadata=}\KeywordTok{read_tsv}\NormalTok{(}\StringTok{"~/Documents/GitHub/bios611-projects-fall-2019-Poutine1025/project_1/data/UMD_Services_Provided_metadata_20190719.tsv"}\NormalTok{)}
\end{Highlighting}
\end{Shaded}

\hypertarget{questions-of-interest}{%
\subsection{Questions of interest}\label{questions-of-interest}}

The basic questions I am interested in are

\begin{itemize}
\tightlist
\item
  What is the relationship among Food Pounds, Clothing Items and Number
  of people in the family for which food was provided?
\item
  How does time influence the amount of food and clothing items?
\end{itemize}

By the analysis of the above questions, I will try to answer some
further questions like

\begin{itemize}
\tightlist
\item
  What is the amount of food need to be provided in 2019?
\end{itemize}

\hypertarget{data-cleaning}{%
\subsection{Data cleaning}\label{data-cleaning}}

\begin{Shaded}
\begin{Highlighting}[]
\KeywordTok{head}\NormalTok{(UMD)}
\end{Highlighting}
\end{Shaded}

\begin{verbatim}
## # A tibble: 6 x 18
##   Date  `Client File Nu~ `Client File Me~ `Bus Tickets (N~ `Notes of Servi~
##   <chr>            <dbl>            <dbl>            <dbl> <chr>           
## 1 1/22~              212                0               NA <NA>            
## 2 1/29~              738                0               NA <NA>            
## 3 1/20~             3455                0               NA <NA>            
## 4 11/2~             1804            21804               NA <NA>            
## 5 12/2~             1806            21806               NA <NA>            
## 6 10/1~             1614            21614               NA financial refer~
## # ... with 13 more variables: `Food Provided for` <dbl>, `Food
## #   Pounds` <dbl>, `Clothing Items` <dbl>, Diapers <dbl>, `School
## #   Kits` <dbl>, `Hygiene Kits` <dbl>, Referrals <chr>, `Financial
## #   Support` <dbl>, `Type of Bill Paid` <chr>, `Payer of Support` <chr>,
## #   Field1 <lgl>, Field2 <lgl>, Field3 <lgl>
\end{verbatim}

The data consists of 79838 observations with 18 variables such as Date,
Client File Number, Food Pounds, Clothing Items and Number of people in
the family for which food was provided. I will foucus on these 5
variables to answer the questions of interest and dicard observations
with NA. For convenience, I will simplify the variable names. Hence,
clean the data as follows:

\begin{Shaded}
\begin{Highlighting}[]
\NormalTok{UMD_selected=UMD }\OperatorTok\StringTok{ }
\StringTok{  }\KeywordTok{select}\NormalTok{(}\StringTok{`}\DataTypeTok{Date}\StringTok{`}\NormalTok{, }\StringTok{`}\DataTypeTok{Client File Number}\StringTok{`}\NormalTok{, }\StringTok{`}\DataTypeTok{Food Pounds}\StringTok{`}\NormalTok{, }\StringTok{`}\DataTypeTok{Clothing Items}\StringTok{`}\NormalTok{, }\StringTok{`}\DataTypeTok{Food Provided for}\StringTok{`}\NormalTok{) }\OperatorTok
\StringTok{  }\KeywordTok{rename}\NormalTok{(}\DataTypeTok{CFN=}\StringTok{`}\DataTypeTok{Client File Number}\StringTok{`}\NormalTok{, }\DataTypeTok{food=}\StringTok{`}\DataTypeTok{Food Pounds}\StringTok{`}\NormalTok{, }\DataTypeTok{clothing=}\StringTok{`}\DataTypeTok{Clothing Items}\StringTok{`}\NormalTok{, }\DataTypeTok{number=}\StringTok{`}\DataTypeTok{Food Provided for}\StringTok{`}\NormalTok{) }\OperatorTok
\StringTok{  }\KeywordTok{drop_na}\NormalTok{()}
\KeywordTok{head}\NormalTok{(UMD_selected)}
\end{Highlighting}
\end{Shaded}

\begin{verbatim}
## # A tibble: 6 x 5
##   Date        CFN  food clothing number
##   <chr>     <dbl> <dbl>    <dbl>  <dbl>
## 1 1/22/2009   212    20        5      3
## 2 1/29/2009   738    25       26      4
## 3 1/20/2009  3455    40       39      6
## 4 2/19/2008  1814     0        0      0
## 5 5/1/1931      4    15       10      1
## 6 1/26/2006     4    10        2      1
\end{verbatim}

Here, ``CFN'' stands for ``Client File Number'', ``food'' stands for
``Food Pounds'', ``clothing'' stands for ``Clothing Items'', and
``number'' stands for ``Number of people in the family for which food
was provided''.

\hypertarget{question-1}{%
\subsection{Question 1}\label{question-1}}

The first question of interest is what the relationship is among Food
Pounds, Clothing Items and Number of people in the family for which food
was provided.

\hypertarget{data-preview}{%
\subsubsection{Data preview}\label{data-preview}}

The histograms of Food Pounds, Clothings Items and Food Provided for are
as follows:

\includegraphics[width=0.3333\linewidth]{project1_files/figure-latex/unnamed-chunk-5-1}
\includegraphics[width=0.3333\linewidth]{project1_files/figure-latex/unnamed-chunk-5-2}
\includegraphics[width=0.3333\linewidth]{project1_files/figure-latex/unnamed-chunk-5-3}

As we can see, there are some extreme values among these 3 variables.
After examining the variables, I decided to discard observations with
Food Pounds larger than 75 or Clothing Items larger than 40 or Food
Provided for larger than 10. Thus, I clean the data and plot histograms
again as follows:

\begin{Shaded}
\begin{Highlighting}[]
\NormalTok{UMD_selected=UMD_selected }\OperatorTok\StringTok{ }\KeywordTok{filter}\NormalTok{((food}\OperatorTok{<}\DecValTok{75}\NormalTok{) }\OperatorTok{&}\StringTok{ }\NormalTok{(clothing}\OperatorTok{<}\DecValTok{40}\NormalTok{) }\OperatorTok{&}\StringTok{ }\NormalTok{(number}\OperatorTok{<}\DecValTok{10}\NormalTok{))}
\end{Highlighting}
\end{Shaded}

\includegraphics[width=0.3333\linewidth]{project1_files/figure-latex/unnamed-chunk-7-1}
\includegraphics[width=0.3333\linewidth]{project1_files/figure-latex/unnamed-chunk-7-2}
\includegraphics[width=0.3333\linewidth]{project1_files/figure-latex/unnamed-chunk-7-3}

We can see after removing the outliers the distributions of these 3
variables become more normal, that is, they in a reasonable range.

\hypertarget{correlation-of-variables}{%
\subsubsection{Correlation of
variables}\label{correlation-of-variables}}

In order to explore the relationship among them, I plotted the
scatterplots for each pair of the variables.

\includegraphics[width=0.3333\linewidth]{project1_files/figure-latex/unnamed-chunk-8-1}
\includegraphics[width=0.3333\linewidth]{project1_files/figure-latex/unnamed-chunk-8-2}
\includegraphics[width=0.3333\linewidth]{project1_files/figure-latex/unnamed-chunk-8-3}

The scatterplots show that Food Pounds is positive correlated with Food
Provided for and there is no apparent relationship in the other two
pairs of variables. Since many records share the same Client File
Number, it is necessary to group the data by Client File Number and see
what happens. The following code groups the data by CFN and creats 2 new
variables - Average Food Pounds and Average Clothing Items.

\begin{Shaded}
\begin{Highlighting}[]
\CommentTok{#subset1}
\NormalTok{UMD_subset1=UMD_selected }\OperatorTok\StringTok{ }
\StringTok{  }\KeywordTok{group_by}\NormalTok{(CFN) }\OperatorTok
\StringTok{  }\KeywordTok{summarize}\NormalTok{(}\DataTypeTok{number=}\KeywordTok{round}\NormalTok{(}\KeywordTok{mean}\NormalTok{(number)), }\DataTypeTok{food=}\KeywordTok{sum}\NormalTok{(food), }\DataTypeTok{clothing=}\KeywordTok{sum}\NormalTok{(clothing), }\DataTypeTok{freq=}\KeywordTok{n}\NormalTok{()) }\OperatorTok
\StringTok{  }\KeywordTok{mutate}\NormalTok{(}\DataTypeTok{food_mean=}\NormalTok{food}\OperatorTok{/}\NormalTok{freq, }\DataTypeTok{clothing_mean=}\NormalTok{clothing}\OperatorTok{/}\NormalTok{freq)}
\end{Highlighting}
\end{Shaded}

After grouping, I plotted the scatterplots for 2 pairs of the variables
and found some correlation.

\begin{Shaded}
\begin{Highlighting}[]
\CommentTok{#number vs food_mean}
\KeywordTok{ggplot}\NormalTok{(}\DataTypeTok{data=}\NormalTok{UMD_subset1,}\KeywordTok{aes}\NormalTok{(}\DataTypeTok{x=}\NormalTok{number, }\DataTypeTok{y=}\NormalTok{food_mean)) }\OperatorTok{+}
\StringTok{  }\KeywordTok{theme_linedraw}\NormalTok{() }\OperatorTok{+}
\StringTok{  }\KeywordTok{geom_point}\NormalTok{() }\OperatorTok{+}
\StringTok{  }\KeywordTok{geom_smooth}\NormalTok{(}\DataTypeTok{method =} \StringTok{"lm"}\NormalTok{, }\DataTypeTok{se =} \OtherTok{TRUE}\NormalTok{) }\OperatorTok{+}
\StringTok{  }\KeywordTok{labs}\NormalTok{(}\DataTypeTok{x=}\StringTok{"Food Provided for"}\NormalTok{,}
       \DataTypeTok{y=}\StringTok{"Average Food Pounds"}\NormalTok{,}
       \DataTypeTok{title =} \StringTok{"Average Food Pounds vs Food Provided for"}\NormalTok{)}
\end{Highlighting}
\end{Shaded}

\includegraphics{project1_files/figure-latex/unnamed-chunk-10-1.pdf}

We can see this figure shows a stronger positive correlation between
Average Food Pounds and Food Provided for.

\begin{Shaded}
\begin{Highlighting}[]
\CommentTok{#food vs clothing}
\KeywordTok{ggplot}\NormalTok{(}\DataTypeTok{data=}\NormalTok{UMD_subset1,}\KeywordTok{aes}\NormalTok{(}\DataTypeTok{x=}\NormalTok{food, }\DataTypeTok{y=}\NormalTok{clothing)) }\OperatorTok{+}
\StringTok{  }\KeywordTok{theme_linedraw}\NormalTok{() }\OperatorTok{+}
\StringTok{  }\KeywordTok{geom_point}\NormalTok{() }\OperatorTok{+}
\StringTok{  }\KeywordTok{geom_smooth}\NormalTok{(}\KeywordTok{aes}\NormalTok{(}\DataTypeTok{colour=}\StringTok{"gam"}\NormalTok{)) }\OperatorTok{+}
\StringTok{  }\KeywordTok{geom_smooth}\NormalTok{(}\DataTypeTok{method =} \StringTok{"lm"}\NormalTok{, }\KeywordTok{aes}\NormalTok{(}\DataTypeTok{colour=}\StringTok{"lm"}\NormalTok{)) }\OperatorTok{+}
\StringTok{  }\KeywordTok{labs}\NormalTok{(}\DataTypeTok{x=}\StringTok{"Food Pounds"}\NormalTok{,}
       \DataTypeTok{y=}\StringTok{"Clothing Items"}\NormalTok{,}
       \DataTypeTok{title=}\StringTok{"Clothing Items vs Food Pounds"}\NormalTok{)}
\end{Highlighting}
\end{Shaded}

\includegraphics{project1_files/figure-latex/unnamed-chunk-11-1.pdf}

After grouping the data, Clothing Items and Food Pounds show a much
stronger positive correlation. Thus, we can fit a linear model to the
data and predict one variable by the other to some degree.

\hypertarget{question-2}{%
\subsection{Question 2}\label{question-2}}

The second question of interest is hows time influences the amount of
food and clothing items. I first extracted the information from variable
Date and remove observations with too small samples size. The records
since 2019 are not complete so I remove them as well.

\begin{Shaded}
\begin{Highlighting}[]
\CommentTok{#subset2}
\NormalTok{UMD_subset2=UMD_selected }\OperatorTok\StringTok{ }
\StringTok{  }\KeywordTok{mutate}\NormalTok{(}\DataTypeTok{Date=}\KeywordTok{as.Date}\NormalTok{(Date, }\DataTypeTok{format=}\StringTok{"%m/%d/%Y"}\NormalTok{)) }\OperatorTok
\StringTok{  }\KeywordTok{mutate}\NormalTok{(}\DataTypeTok{year=}\KeywordTok{year}\NormalTok{(Date), }\DataTypeTok{month=}\KeywordTok{month}\NormalTok{(Date), }\DataTypeTok{day=}\KeywordTok{day}\NormalTok{(Date)) }\OperatorTok
\StringTok{  }\KeywordTok{mutate}\NormalTok{(}\DataTypeTok{month=}\KeywordTok{as.factor}\NormalTok{(month), }\DataTypeTok{day=}\KeywordTok{as.factor}\NormalTok{(day)) }\OperatorTok
\StringTok{  }\KeywordTok{filter}\NormalTok{((year}\OperatorTok{>}\DecValTok{2004}\NormalTok{) }\OperatorTok{&}\StringTok{ }\NormalTok{(year}\OperatorTok{<}\DecValTok{2019}\NormalTok{))}
\end{Highlighting}
\end{Shaded}

Then we can plot the bar chart of Year and examine the distribution
across months.

\begin{Shaded}
\begin{Highlighting}[]
\CommentTok{#bar chart of year}
\KeywordTok{ggplot}\NormalTok{(}\DataTypeTok{data=}\NormalTok{UMD_subset2,}\KeywordTok{aes}\NormalTok{(year)) }\OperatorTok{+}
\StringTok{  }\KeywordTok{geom_bar}\NormalTok{(}\KeywordTok{aes}\NormalTok{(}\DataTypeTok{fill=}\NormalTok{month)) }\OperatorTok{+}
\StringTok{  }\KeywordTok{labs}\NormalTok{(}\DataTypeTok{x=}\StringTok{"Year"}\NormalTok{,}
       \DataTypeTok{title =} \StringTok{"Bar chart of Year"}\NormalTok{)}
\end{Highlighting}
\end{Shaded}

\includegraphics{project1_files/figure-latex/unnamed-chunk-13-1.pdf}

We can see that the amount of records is increasing by year and they are
distributed uniformly across months. The following two boxplots show the
distributions of Food Pounds and Clothing Items across months.

\includegraphics[width=0.5\linewidth]{project1_files/figure-latex/unnamed-chunk-14-1}
\includegraphics[width=0.5\linewidth]{project1_files/figure-latex/unnamed-chunk-14-2}

These two figures indicate the distributions of Food Pounds and Clothing
Items are roughly the same across the months.

In order to predict the amount of Food Pounds of 2019, it is meaningful
to plot the scatterplot of Food Pounds and Year.

\begin{Shaded}
\begin{Highlighting}[]
\CommentTok{#subset3}
\NormalTok{UMD_subset3=UMD_subset2 }\OperatorTok\StringTok{ }
\StringTok{  }\KeywordTok{group_by}\NormalTok{(year) }\OperatorTok
\StringTok{  }\KeywordTok{summarize}\NormalTok{(}\DataTypeTok{food=}\KeywordTok{sum}\NormalTok{(food), }\DataTypeTok{clothing=}\KeywordTok{sum}\NormalTok{(clothing), }\DataTypeTok{number=}\KeywordTok{sum}\NormalTok{(number), }\DataTypeTok{freq=}\KeywordTok{n}\NormalTok{())}

\KeywordTok{ggplot}\NormalTok{(}\DataTypeTok{data =}\NormalTok{ UMD_subset3, }\KeywordTok{aes}\NormalTok{(}\DataTypeTok{x=}\NormalTok{year, }\DataTypeTok{y=}\NormalTok{food)) }\OperatorTok{+}
\StringTok{  }\KeywordTok{geom_point}\NormalTok{() }\OperatorTok{+}
\StringTok{  }\KeywordTok{geom_smooth}\NormalTok{(}\DataTypeTok{method =} \StringTok{"lm"}\NormalTok{) }\OperatorTok{+}
\StringTok{  }\KeywordTok{labs}\NormalTok{(}\DataTypeTok{x=}\StringTok{"Year"}\NormalTok{,}
       \DataTypeTok{y=}\StringTok{"Food Pounds"}\NormalTok{,}
       \DataTypeTok{title=}\StringTok{"Food Pounds vs Year from 2005 to 2018"}\NormalTok{)}
\end{Highlighting}
\end{Shaded}

\includegraphics{project1_files/figure-latex/unnamed-chunk-15-1.pdf}

It was found that there seemed to be a linear relationship between Food
Pounds and Year. Thus, we can predict the amount of Food Pounds by
fitting a linear model to the data.

\hypertarget{question-3}{%
\subsection{Question 3}\label{question-3}}

The last question is what the amount of food need to be provided is in
2019. I used the figure ``Food Pounds vs Year from 2005 to 2018'' to
answer question 3. By fitting a linear model to the corresponding data,
I found the predicted value.

\begin{Shaded}
\begin{Highlighting}[]
\NormalTok{model_FoodvsYear=}\KeywordTok{lm}\NormalTok{(food}\OperatorTok{~}\NormalTok{year, }\DataTypeTok{data =}\NormalTok{ UMD_subset3)}
\NormalTok{Food2019=}\KeywordTok{predict}\NormalTok{(model_FoodvsYear, }\DataTypeTok{newdata =} \KeywordTok{data.frame}\NormalTok{(}\DataTypeTok{year=}\DecValTok{2019}\NormalTok{))}
\KeywordTok{print}\NormalTok{(}\KeywordTok{paste}\NormalTok{(}\StringTok{"The predicted value of Food Pounds of 2019 is"}\NormalTok{, Food2019))}
\end{Highlighting}
\end{Shaded}

\begin{verbatim}
## [1] "The predicted value of Food Pounds of 2019 is 157291.472527474"
\end{verbatim}

\hypertarget{conclusion}{%
\subsection{Conclusion}\label{conclusion}}

\begin{itemize}
\tightlist
\item
  There is a stronger positive correlation between Average Food Pounds
  and Food Provided for.
\item
  There is a stronger positive correlation between Food Pounds and
  Clothing Items.
\item
  The amount of records is increasing by year and they are distributed
  uniformly across months.
\item
  The distributions of Food Pounds and Clothing Items are roughly the
  same across the months.
\item
  There is a linear relationship between Food Pounds and Year.
\item
  The predicted value of Food Pounds of 2019 is around 157291.
\end{itemize}


\end{document}
